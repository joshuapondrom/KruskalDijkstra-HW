\documentclass{article}
\usepackage{indentfirst}
\usepackage{tabto}
\usepackage{listings}

\begin{document}
\title{Minimum Spanning Trees \& Shortest Path Trees\\\large Not a paper about arborology}
\author{Joshua Pondrom}
\maketitle

\section{Motivation}
Say there is a problem in which we need to connect everything using only
certains paths, such as networking, electrical grids, or trying to go to
multiple places in a minimal amount of time. These are all problems that
can be reduced down to minimum spanning trees, or to shortest path
trees then formally and 
mathematically solved using graph algorithms. Like many graph algorithms,
minimum spanning trees and shortest path trees, are very applicable to 
real, tangible problems found in our everyday life.
\section{Background}
Formally a minimum spanning tree is defined on a weighted graph, G(v, e), such that
all vertices, v, are connected through edges, e, and in a way which
produces a acyclic tree with a minimum possible total weight of edges in
the tree. The edge weights can represent anything which is applicable to
the problem, such as distance, resistance, cost, \textit{et cetera}.

Likewise, a shortest path tree is defined on a weighted graph with a source, 
s. The shortest path tree will be a tree branching from s connecting the
source to every vertex in the graph with a minimum total weight.

Problems like this have been being worked on since around the 1920's, such as
Boruvka's algorithm for minimum spanning trees developed in 1926. Later there
was algorithms developed by Jarnik (1930), Prim (1957), and Dijkstra (1959).
More recently there was improvements to the algorithm made by Karger, Klein,
and Tarjan (1995).
\section{Pseudocode}
There will be pseudocode and implementation of both Kruskal's and Dijkstra's
algorithm for minimum spanning trees.

First, there will be Kruskal's algorithm. This algorithm was first seen in
1956 and written by Joseph Kruskal, American mathematician and computer
scientist. The algorithm uses greedy techniques but will always give a
correct answer. In simple terms what the algorithm does is sort the edge
lengths, takes the minimum edge if it satisfies the tree properties, and
repeats those steps until there is a minimum spanning tree. The complexity
for this algorithm is $O(|E|log|E|)$.

Last but definitely least, we have Dijkstra's algorithm. Dijkstra's algorithm
is not exactly the same as Kruskal's. Dijkstra's gives the shortest path
between a source vertex and all the rest of the vertices in a graph. This was 
a algorithm published in 1959 by famous Dutch computer scientist, Edsger W.
Dijkstra. In simple terms the way this algorithm works is as follows; Start with
the source. Let distance to all other vertices to be infinite, look at all
neighbors of source and if the distance through this path is less that what is
stored, save it was the new path and distance to that vertex. Repeat and keep
placing the newly visited nodes in the queue. Continue until the queue is
empty. The complexity for this algorithm is $O(|E| + |V|log|V|)$.



The pseudocode is as follows; for Kruskal and Dijkstra respectively.
\begin{lstlisting}
Kruskal(Graph G):
	A = Empty Set
	P = Partition of vertices of G
	E = Sorted edges of G, ascending
	While E is not empty
		#Invariant : A should satisfy tree 
		#            properties
		(u, v) = E.pop
		#Invariant : Edge (u, v) should be 
		#            minimum edge remaining
		if P.find(u) does not equal P.find(v)
			A = A Union (u,v)
			#Invariant : A contains 
			#            minimum trees
	#Invariant : A is a minimum spanning tree on G
	return A
\end{lstlisting}
\newpage
\begin{lstlisting}
Dijkstra(Graph G, Source S):
	A = Empty Set
	For each vertex in G
		distance = Infinity
		previous = Undefined
		A = A Union vertex
	distance to S = 0
	While A is not empty
		#Invariant : all vertices in A are connected to our
		#            current 
		v = vertex in A with minimum distance
		remove v from A
		for each vertex connected to v, u
			#Invariant : edge(v, u) union A should
			#            maintain tree property
			new = distance to u + length(u,v)
			if new < distance to v
				distance to v = new
				previous to v = u
			#Invariant : edge(v, u) should be the
			#            shortest edge to connect
			#            v and u seen so far
	return A
\end{lstlisting}
\section{Testing Plan}
For the program, Javascript will be good for a implementation language.
The reason for this is the abundance of libraries for easily showing graphs
and the simplicity of the language, allowing for more focus on the algorithm.
Additionally it should be easy to make to make a interface to present the 
code with, with functionality like interacting with the graph and showing
changes. The program can be timed by using the GNU Unix tool \textit{time}.
This is a simple lightweight progrgam that can be used to see how much time
the program uses. It can be combined with a small shell script if there is
more data needed, such as a dataset to get an average.
\subsection{Invariants}
Luckily, Javascript has a assert function to help code invariants into the 
code and guarantee code correctness. In Javascript we can also give a 
description to assert statements to better see where any possible errors in
the code are. There can also be function definitions to put in assert statements
to help do more verbose and powerful invariants.
\section{Testing}
The data gathered by timing the code on data sets is as follows:
\begin{center}
\begin{tabular}{||c | c c c||}
	\hline
	trial & 1 & 2 & 3\\
	\hline\hline
	Kruskal&.11s&.09s&.10s\\
	\hline
	Dijkstra&.12s&13s&10s\\
	\hline
\end{tabular}
\end{center}
Overall, they were close in runtime with Dijkstra's algorithm taking slightly
more time on average. This makes sense because the runtime of Dijkstra's 
algorithm is $O(|V|^{2})$ while Kruskal is $O(|E|log|V|)$
\section{Problems Encountered}
The main problem I encountered when implementing the algorithms was the
language. I started out with javascript because I personally thought it 
would be easy to give a nice visualization of the graph, however, in 
practice nothing was working out. After some time I went to python and
got the programs to work smoothly. The only problem I had while implementing
in python was the making of a graph class. Dijkstra's and Kruskal's ended up
needing slightly different graph classes for a clean implementation so I split
it into two programs
\section{Conclusion}
In the end, both algorithms are quintessential graph algorithms that have
many applications and purposes in the world of computer science. Dijkstra's
algorithm tends to take the spotlight due to being more applicable to 
common problems people are trying to solve, but both algorithms should be
considered as building blocks for solving more complicated graph problems.
\end{document}
